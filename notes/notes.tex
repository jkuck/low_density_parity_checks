\documentclass[11pt]{article}
\usepackage{amsmath, amssymb, amsfonts, amsthm}
\usepackage[top=1in, bottom=1in, left=1in, right=1in]{geometry}
\usepackage{enumerate}
\usepackage{enumitem}
\usepackage{mathtools}
\usepackage{todonotes}

\allowdisplaybreaks[2]

\renewcommand{\d}{\,d}

\DeclareMathOperator*{\argmin}{arg\,min}
\DeclareMathOperator*{\argmax}{arg\,max}
\DeclareMathOperator{\dist}{dist}
\DeclareMathOperator{\diam}{diam}
\DeclareMathOperator*{\esssup}{ess\,sup}

% \DeclareMathOperator{\p}{\mathbf{Pr}}
\DeclareMathOperator{\p}{Pr}
\DeclareMathOperator{\E}{\mathbf{E}}
\DeclareMathOperator{\var}{\mathbf{Var}}
% \DeclareMathOperator{\cov}{\mathbf{Cov}}
\DeclareMathOperator{\cov}{cov}
\DeclareMathOperator{\diag}{\mathbf{diag}}
% \DeclareMathOperator{\p}{Pr}
% \DeclareMathOperator{\E}{\mathbb{E}}
\DeclareMathOperator{\1}{\mathbf{1}}

\providecommand{\abs}[1]{\left\lvert#1\right\rvert}
\providecommand{\norm}[1]{\left\lVert#1\right\rVert}

\newcommand\numberthis{\addtocounter{equation}{1}\tag{\theequation}}

\newtheorem{theorem}{Theorem}
\newtheorem{lemma}{Lemma}
\newtheorem{corollary}{Corollary}
\newtheorem{definition}{Definition}

\bibliographystyle{unsrt}

\title{SAT model counting with LDPC codes - Research notes}
\author{Tri Dao -- \texttt{trid@stanford.edu}}
\date{\today}
\begin{document}

\maketitle

\section{Parity constraints with LDPC codes}
\label{sec:parity_constraints}

\subsection{First and second moments of projected partition function with
  low-density parity constraints}
\label{subsec:first_second_moments}

We re-derive the mean and variance of the estimators in the original WISH papers
\cite{wishicml13,ermon2014low} for general weights.

Suppose we have a graphical model with variable $X \in \mathcal{X} = \{
0, 1 \}^n$:
\begin{equation*}
  p(x) = \frac{1}{Z} \phi(x), \qquad
  Z = \sum_{x \in \mathcal{X}} \phi(x),
\end{equation*}
where $\phi(x) \geq 0$ is the factor/potential at each configuration $x$.
We are interested in approximating $Z$.

We introduce $m$ parity constraints by generating a random matrix $A \in \{ 0, 1
\}^{m \times n}$ and a random vector $b \in \{ 0, 1 \}^m$ where $A_{ij}
\overset{\text{iid}}{\sim} \mathrm{Ber}(f)$ and independently $b_i
\overset{\text{iid}}{\sim} \mathrm{Ber}(1/2)$.
Suppose that $0 < f \leq 1/2$.
Let
\begin{equation*}
  Z(A, b) = \sum_{x \colon Ax = b \bmod 2} \phi(x).
\end{equation*}

We first compute the first moment:
\begin{align*}
  \E[Z(A, b)]
  &= \E\left[\sum_{x} \phi(x) I \{ Ax = b \}\right] = \sum_{x} \phi(x) \Pr[Ax =
    b] \\
  &= \sum_{A} \Pr(A) \sum_{x} \phi(x) \Pr[b = Ax \mid A] = \sum_{A} \Pr(A)
    \sum_{x} \phi(x) 2^{-m} \\
  &= 2^{-m} \sum_{x} \phi(x) = 2^{-m} Z.
\end{align*}
Hence $2^m Z(A, b)$ is an unbiased estimator of $Z$.
Note that this holds for any distribution of $A$, as long as $b_i
\overset{\text{iid}}{\sim} \mathrm{Ber}(1/2)$.

Now we compute the second moment:
\begin{align*}
  \E[Z(A, b)^2]
  &= \sum_{x, x'} \phi(x) \phi(x') \E \left[ I \{ Ax = b \} I \{ Ax' = b \}
    \right] \\
  &= \sum_{x} \phi(x)^2 \Pr[Ax = b] + \sum_{x \neq x'} \phi(x) \phi(x') \Pr[Ax =
    Ax' = b] \\
  &= 2^{-m} \left[ \sum_{x} \phi(x)^2 + \sum_{x \neq x'} \phi(x) \phi(x')
    \Pr[A(x - x') = 0] \right].
    \numberthis \label{eq:second_moment}
\end{align*}
Let $H(x, x')$ be the Hamming distance between $x$ and $x'$.
If $H(x, x') = h > 0$ then $A(x - x') = 0$ if and only if the sum of $h$ columns
of $A$ yields $0 \pmod 2$ (it does not matter which columns).
Since all the rows of $A$ are independent, we compute the probability of the sum
of $h$ entries from one row being 0.
This is equivalent starting at 0, taking $h$ steps of a random walk with
probability $f$ of switching from 0 to 1 and vice versa, and probability $1 - f$
of staying in place, and ending up at state 0.
If we start at state 0 at time 0, after $h$ time steps, the probability of being
in state 0 is $\frac{1 + (1 - 2f)^h}{2}$.
Hence $\Pr[A(x - x') = 0] = \frac{\left(1 + (1 - 2f)^{H(x, x')}\right)^{m}}{2^m}$.
Substituting this in equation~\eqref{eq:second_moment} gives
\begin{equation*}
  \E[Z(A, b)^2]
  = 2^{-m} \left[ \sum_x \phi(x)^2 + 2^{-m} \sum_{x \neq x'} \phi(x) \phi(x')
    \left( 1 + (1 - 2f)^{H(x, x')} \right)^m \right].
\end{equation*}

\paragraph{Model counting.}
Suppose that $\phi(x) = I \{ x \in S \}$ for some set $S$.
Then $\phi(x) \in \{ 0, 1 \}$ for all $x$, so $\phi(x)^2 = \phi(x)$.
Since $Z = \abs{S}$,  so $\E[Z(A, b)] = 2^{-m} \abs{S}$.
We can also simplify the second moment in equation~\ref{eq:second_moment}:
\begin{align*}
  \var[Z(A, b)] = 2^{-m} \left[ \abs{S} + 2^{-m} \sum_{\substack{x, x' \in S \\ x \neq x'}} \left( 1 + (1 - 2f)^{H(x, x')} \right)^m \right].
\end{align*}
After a few more steps of simplification (which we will not repeat here), we
obtain the bound in \cite{ermon2014low}.

\subsection{Hashing with LDPC matrices - Loose bounds}

Now we will change the distribution of matrix $A$ to be closer to LDPC matrices.
This section serves as a warm-up as we will only prove a loose bound that the
variance is no worse than the previous i.i.d.\ case.

As an example, for the $2 \times 3$ case, we may want $A_{ij} \sim \mathrm{Ber}(F_{ij})$
where
\begin{equation*}
  F = \begin{bmatrix} 1 & f & f \\ f & 1 & f \end{bmatrix}.
\end{equation*}
That is, we may want to have a diagonal of 1's.
Moreover, we may even randomly permute the columns of $A$.
Note that this makes the rows of $A$ no longer independent: knowing that $A_{12}
= 1$ means that $A_{12}$ was probably sampled from $\mathrm{Ber}(1)$, so
$A_{22}$ is probably sampled from $\mathrm{Ber}(f)$, which means it is less
likely than usual to be 1.

To ensure the analysis still go through (easily), we will make some compromises:
\begin{itemize}
  \item We will not sample from $\mathrm{Ber}(1)$ (surely 1) but from
  $\mathrm{Ber}(1 - f)$ (1 with high probability) instead.
  \item We assume that the indices having $\mathrm{Ber}(1-f)$ distribution are
  chosen without ``peeking'' at the values of the other entries.
  We will make this precise soon.
\end{itemize}

From the first assumption, we have $A_{ij} \sim \mathrm{Ber}(F_{ij})$ with $F_{ij}
\in \{ f, 1 - f \}$.
Since $\mathrm{Ber}(1 - f) = \mathrm{Ber}(1) + \mathrm{Ber}(f)$ (addition is
done modulo 2),
We can write this as $A = A' + A''$, where $A'_{ij} \sim \mathrm{Ber}(f)$ and
$A''_{ij}$ is either 0 (where $A_{ij}$ is supposed to be $\mathrm{Ber}(f)$) or 1
(where $A_{ij}$ is supposed to be $\mathrm{Ber}(1 - f)$).
Note that $A''$ is also random, and the second assumption above corresponds to
assuming that $A'$ and $A''$ are independent.

This decomposition $A = A' + A''$ corresponds to the following sampling
procedure: we sample $A''$ to determine which $A_{ij}$ will start with 1 while
the rest start with 0, then we sample the coin flips $A'$ to decide which
$A_{ij}$ will get flipped from 1 to 0 or 0 to 1.
Note that $A''$ can model the random permutations of columns of $A$, as long as
the permutation doesn't depend on the coin flips in $A'$.

Surprisingly, this decomposition allows the analysis to go through almost
verbatim, even if the rows of $A$ are no longer independent.
The crucial quantity is $\Pr[A(x - x') = 0]$:
\begin{align*}
  \Pr[A(x - x') = 0]
  &= \Pr[(A' + A'')(x - x') = 0] \\
  &= \Pr[A'(x - x') = A''(x - x')] \\
  &= \sum_{\delta} \Pr[A''(x - x') = \delta] \Pr[A'(x - x') = \delta],
\end{align*}
where we get the last equality from the independence of $A'$ and $A''$.
Recall from Section~\ref{subsec:first_second_moments} that $\Pr[(A'(x - x'))_i =
0]$ is the probability of a random walk of length $H(x, x')$ starting at 0 and
ending at 0.
Now $\Pr[(A'(x - x'))_i = \delta_i]$ is the probability of a random walk of length
$H(x, x')$ starting at 0 and ending at $\delta_i$ (could be 0 or 1).
Since $f \leq 1/2$, this probability is less than that of ending at 0, so $\Pr[A'(x -
x') = \delta] \leq \Pr[A'(x - x') = 0]$.
Therefore
\begin{equation*}
  \Pr[A(x - x') = 0] \leq \sum_{\delta} \Pr[A''(x - x') = \delta] \Pr[A'(x - x') = 0] = \Pr[A'(x
  - x') = 0].
\end{equation*}
We have thus reduced to the i.i.d.\ situation in Section~\ref{subsec:first_second_moments}.
Thus the same variance bound still holds.

\subsection{Tighter bound for block size 1}
\label{subsec:block_size_1}

Now we prove a tighter bound for a special case with block size $k = 1$ and some
entries are fixed to be 1 deterministically (not sampled from $\mathrm{Ber}(1 -
f)$).
The sampling procedure for $A$ is as follows.
First set matrix $A''$ of size $m \times n$ to have all 1 on its diagonal:
\begin{equation*}
  A'' = 
  \begin{bmatrix} 
    1 & 0 & \hdots & 0 & 0 & \hdots & 0 \\ 
    0 & 1 & \hdots & 0 & 0 & \hdots & 0 \\
    \vdots & \vdots & \ddots & \vdots & & \hdots & \\
    0 & 0 & \hdots & 1 & 0 & \hdots & 0
  \end{bmatrix}
\end{equation*}
Then we flip each zero entry of $A''$ to one independently with probability $f$
to obtain $A$.

Fix some $x \in S$.
The probability we care about is $\sum_{x' \in S, x' \neq x} \Pr[A(x - x') = 0]$.
Let $w = H(x, x')$ be the Hamming distance and let $w^*$ be the largest possible
distance.
For each row of $A''$, either the 1 entry will coincide with the 1 entries in $x
- x'$ or not.
If it coincides, the random walk starts with 1 instead of 0 and the length is
$w-1$, so its probability is $r^{w-1, f}(1, 0)$, which we can show to be less
than $r^{w, f}(0, 0)$. 
\todo{Prove this.}
If it doesn't coincide, then the random walk still starts with 0 and take $w$
steps, so the probability is $r^{w, f}(0, 1)$.
Note that the rows of $A$ are still independent so $\Pr[A(x - x') = 0]$ is still
a product of such random walk probabilities.
For $1 \leq w \leq n - m$, it is possible in the worst case for the random walk to
always start from 0 (i.e., $x - x'$ is 1 only on the last $n - m$ coordinates).
Another way to phrase this is that the worst case is when there are $2^{n-m}-1$
points that packed in a hypercube around $x$.
But for $w \geq n - m + 1$, $x - x'$ must have $w - (n - m)$ entries that are one
in the first $m$ coordinates, and so $w - (n - m)$ random walks starts from 1
instead of 0.
Thus we can bound
\begin{equation*}
  \sum_{x' \in S, x' \neq x} \Pr[A(x - x') = 0] \leq \sum_{w=1}^{n - m} \binom{n}{w} [r^{(w,
    f)}(0, 0)]^m + \sum_{w=n-m+1}^{w^*} \binom{n}{w} [r^{(w-1, f)}(1, 0)]^{w-(n-m)} [r^{(w, f)}(0, 0)]^{n-m}.
\end{equation*}

\subsection{Tighter bound for block size 1 with permutation}
\label{subsec:block_size_1_perm}

Now we analyze a variant of the previous sampling procedure, where we randomly
``permute'' the columns of $A''$.

\paragraph{Sampling with replacement.}
We first analyze the case where each rows of $A''$ are formed independently:
uniformly choose an index from 1 to $n$ and set to 1.
That is, for each row, choose the index to set to 1 by sampling from $\{ 1,
\dots, n \}$ \emph{with} replacement.

The rows of $A$ are still independent.
How does this change the random walk?
Since $x - x'$ has weight $w$, there's a $w/n$ chance that the index picked in
$A''$ will hit $x - x'$, and there's a $1 - w/n$ chance that it will not.
Then the random walk starts with 1 with probability $w/n$, and starts with 0
with probability $1-w/n$.
This is the same for all rows.
Therefore
\begin{equation}
  \sum_{x' \in S, x' \neq x} \Pr[A(x - x') = 0] \leq \sum_{w=1}^{w^*} \binom{n}{w} \left(
    \frac{w}{n} r^{(w-1, f)}(1, 0) + \frac{n-w}{n} r^{(w, f)}(0, 0)\right)^m. 
  \label{eq:k_1_with_replacement}
\end{equation}
This looks simpler already.
\todo{Careful, make sure that the inner function is monotonic wrt $w$, otherwise
the worst-case set isn't a Hamming ball.}
However, because of AM-GM inequality, I suspect this is slightly worse than the
bound with true permutation: we always multiply a bunch of random walk
probabilities together, so if these probabilities are the same then the product
will be large.

\paragraph{Sampling without replacement.}
The rows of $A''$ are no longer independent: uniformly choose an index from 1 to
$n$ that has not been chosen by earlier rows and set it to 1.
That is, for each row, choose the index to set to 1 by sampling from $\{ 1,
\dots, n \}$ \emph{without} replacement.
Let $C = \{ j \colon \exists i \text{ such that } A''_{i, j} = 1 \}$ be the set of
column indices chosen.

Note that after conditioning on the permutation (i.e., conditioning on matrix
$A''$), the rows of $A$ are independent.
For a fixed $x, x'$, let $I = \{ i \colon (x - x')_i = 1 \}$ be the set of
indices where $x - x'$ is 1.
The random walk probability depends on how many of the column indices chosen
fall into $I$.
If $w' = \abs{C \cap I}$ then $0 \leq w' \leq \min(m, w)$ and out of the $m$ random walks
corresponding to $m$ rows, $w'$ of them start with 1 and take $w - 1$ steps,
while the rest start with 0 and take $w$ steps.

Since the probability doesn't depend on the order of column indices selected in
the sampling procedure of $A''$, we can ignore the order in subsequent
calculation.
The problem boils down to computing the number of permutations that hit $I$ in
$w'$ places.
The sampling procedure can be rephrased as follows. 
First choose $w'$ out of $w$ indices in $I$ to be hit ($\binom{w}{w'}$ choices).
Then we need to place the remaining $m - w'$ rows into $n - w$ indices (that are
not in $I$), so there are $\binom{n - w}{m - w'}$ choices.
Thus then number of choices is $\binom{w}{w'} \binom{n - w}{m - w'}$.
Since there are $\binom{n}{m}$ permutations in total, we obtain the bound
\begin{equation}
  \sum_{x' \in S, x' \neq x} \Pr[A(x - x') = 0] \leq \sum_{w=1}^{w^*} \binom{n}{w}
  \sum_{w'=0}^{\min(m, w)}\frac{\binom{w}{w'} \binom{n - w}{m - w'}}{\binom{n}{m}}
  [r^{(w-1, f)}(1, 0)]^{w'} [r^{(w, f)}(0, 0)]^{m-w'}.  
  \label{eq:k_1_without_replacment}
\end{equation}

\subsection{The case of larger block size}
\label{subsec:block_size_larger}

What if in each row of $A''$ we have more than one 1 entries (i.e., block size
larger than 1)?
When will that improve the bound?

Analyzing this is difficult as we have to keep track of how many rows hit the
set $I$ at how many places, since that determines whether the random walk starts
with 0 or 1 and the length of the walk.
I believe it's impossible to compute this in polynomial time, since we need to
keep track of and exponential number of such probabilities corresponding to an
exponential number of random walk length combinations.
Instead, we will make a simplifying assumption that makes the analysis much
easier: after sampling the skeleton matrix $A''$, we flip all entries each with
probability $f$ (not just the zero entries) to get $A$.
Then the random walk always has length $w$.

\paragraph{Sampling with replacement.}
Again, we first consider the simple case where each rows of $A''$ are sampled
independently: $k$ indices are picked at random (without replacement) to be set
to 1.
The probability that theses $k$ indices hit the set $I = \{ i \colon (x - x')_i
= 1 \}$ at $t$ places (for $0 \leq t \leq \min(k, w)$) is $\frac{\binom{w}{t} \binom{n
    - w}{k - t}}{\binom{n}{k}}$, by exactly the same reasoning as in the
previous section (sampling without replacement). 
If $t$ is odd, the random walk starts with 1, and if $t$ is even the random walk
starts with 0.
Therefore
\begin{equation*}
  \sum_{x' \in S, x' \neq x} \Pr[A(x - x') = 0] \leq \sum_{w=1}^{w^*} \binom{n}{w} \left(
    p_\mathrm{odd}(n, w, k) r^{(w, f)}(1, 0) + p_\mathrm{even}(n, w, k)
    r^{(w, f)}(0, 0)\right)^m, 
  \label{eq:k_large_with_replacement}
\end{equation*}
where
\begin{equation*}
  p_\mathrm{odd}(n, w, k) = \sum_{\substack{0 \leq t \leq \min(k, w) \\ t \text{ odd}}} \frac{\binom{w}{t}
        \binom{n-w}{k-t}}{\binom{n}{k}}, \qquad
  p_\mathrm{even}(n, w, k) = \sum_{\substack{0 \leq t \leq \min(k, w) \\ t \text{ even}}} \frac{\binom{w}{t}
        \binom{n-w}{k-t}}{\binom{n}{k}}.
\end{equation*}
If $k = 1$ then $p_\mathrm{odd}(n, w, k) = \frac{w}{n}$ and
$p_\mathrm{even}(n, w, k) = \frac{n - w}{n}$, so we recover the formula~\eqref{eq:k_1_with_replacement}.
As $k$ increases, $p_\mathrm{odd}$ and $p_\mathrm{even}$ get closer to $0.5$, so
this helps for small $w$ and hurts for large $w$.
But the difference between $r^{(w, f)}(1, 0)$ and $r^{(w, f)}(0, 0)$ is big for
small $w$ and small for big $w$.
Thus increasing $k$ will decrease the variance.

\paragraph{Sampling without replacement.}
This case is a little annoying and I don't think we can get a closed form
expression.
Nevertheless, we can give a recursive formula to compute the bound in polynomial
time\footnote{Recall that we'll be calling SAT solvers so it's essentially
  ``free'' to run some polynomial time procedure.}.
The sampling procedure for $A''$ is as follows: for each row, pick $k$ indices
(without replacement) at random among those that have not been selected by
previous rows.
We require that $k \leq n / m$.

Let $C_1, \dots, C_m$ be the set of indices selected in each row, and let $C =
\bigcup_{i=1}^m C_i$.
If $\abs{C_i \cap I}$ is odd then the random walk in that row starts at 1, and if
$\abs{C_i \cap I}$ is even that it starts at 0.
We care about the number of rows that hit $I$ an odd number of times.

Let $w' = \abs{C \cap I}$ be the number of slots in $I$ that are hit by any of the
rows, then $0 \leq w' \leq \min(w, mk)$.
We can rephrase the sampling procedure as follows: first choose a random subset
of $\{ 1, \dots, n \}$ of size $mk$, then randomly order the indices in this
subset, and finally put them in blocks where $C_1$ contains the first $k$
indices, $C_2$ contains the next $k$ indices, and so on.
There are $(mk)! 
\binom{n}{mk} / (k!)^m$ permutations (we divide by $(k!)^m$ because we ignore
the order within each row), and if we know that $C$ hits $I$ at $w'$ places then
there are $\binom{w}{w'} \binom{n - w}{mk - w'}$ ways to choose such subsets.
The central question is, after permuting the $mk$ indices in $C$ and putting
them into blocks $C_1, \dots, C_m$, how many of such $(mk)!$ permutations result
in exactly $q$ blocks that hit $I$ an odd number of times, with $0 \leq q \leq
\min(w', m)$.

Let $f(r, s, t)$ be the number of permutations of $rk$ indices, with $s$ in $I$
and $rk - s$ in $I^C = \{ 1, \dots, n \} \setminus I$, such that after putting them into
$r$ blocks, $t$ of them hit $I$ an odd number of times.
The number we care about is $f(m, w', q)$.
We can compute $f(r, s, t)$ recursively.
Let $0 \leq h \leq \min(s, k)$ be the number of times the first block will hit $I$.
Now we choose $k$ indices out of $rk$ such that $h$ of them belong to the $s$
slots in $I$, so there are $\binom{s}{h} \binom{rk - s}{k - h}$ choices (recall
that we ignore the order within each row).
The problem reduces, since we have $r - 1$ rows left, with $s - h$ slots.
Thus
\begin{equation*}
  f(r, s, t) = \sum_{0 \leq h \leq \min(s, k)} \binom{s}{h}
  \binom{rk-s}{k - h} f(r-1, s-h,
  t-\mathbb{I}(h \text{ odd})) 
\end{equation*}
For the base case, $f(0, s, 0) = 1$ and $f(0, s, t) = 0$ for $t > 1$, $f(r, s,
-1) = 0$, and $f(r, s, t) = 0$ if $rk < s$.
Notice that we can compute $f(m, w', q)$ in polynomial time by dynamic
programming.

After computing $f(m, w', q)$, we obtain the bound
\begin{equation*}
  \sum_{x' \in S, x' \neq x} \Pr[A(x - x') = 0] \leq \sum_{w=1}^{w^*} \binom{n}{w} \sum_{w'=0}^{\min(w, mk)}
  \sum_{q=0}^{\min(w', m)} \frac{\binom{w}{w'} \binom{n - w}{mk-w'} f(m, w',
    q)}{(mk)! \binom{n}{mk} / (k!)^m} [r^{(w, f)}(1, 0)]^q [r^{(w, f)}(0, 0)]^{m-q}. 
  \label{eq:k_large_with_replacement}
\end{equation*}

\bibliography{notes}

\end{document}
